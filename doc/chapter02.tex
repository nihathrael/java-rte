\section{Building our own RTE system}
In this section we describe our best RTE system.

\subsection{Considerations from the prior systems}
All prior systems turned out to be worse than the normal lemma matching, Which to this point was still our  best system
with a score of 63.25\%. We did not see any big potential in improving the basic lemma matching further, so we decided
to try to tune our machine learning matcher \textit{BasicMahoutMatcher} in the form of \textit{MahoutMatcher}. Our basic
idea was to use the best non-machine-learner systems as features for the machine learner, hoping this would give the
machine learner a good basis to work with.

\subsection{Implementation}
We already had the basic \textit{BasicMahoutMatcher} from part three, which we decided to tune. The base of this matcher
was the \textit{OnlineLogisticRegression} learning algorithm distributed with the Mahout machine-learning library. We
played around with all our basic matchers as features, but ended up only using a quite small subset in the final
version. Only 6 features were used:
\begin{itemize}
    \item Lemma Matching
    \item IDF Lemma Matching
    \item Lemma+POS Matching
    \item BleuScore
    \item Polarity
    \item WordNet Synonym Matching
\end{itemize}

For all features for which we had matchers (all except Polarity) we used the matchers estimate whether a text/hypothesis
pair was entailing or not as value, as this is already a convinient value between 0 and 1. All matchers except for
BleuScore already contain the Polarity measurement already as a sort of "death" criteria, where we set the estimate to
0 if the polarity doesn't match. We still use it as seperate feature, by setting the value to 1 if it matches and 0 if
it doesn't, this proved helpful (without polarity feature, the result was ~62.6\%).

As we were not happy with the results we were able to achieve with our basic set of matchers, we decided to implement
the WordNet library using JWI\footnote{JWI website:\url{http://projects.csail.mit.edu/jwi}} as access to WordNet and the
JavaSimLibrary\footnote{JavaSimLibrary website: \url{http://pertomed.spim.jussieu.fr/~lma/jsl/}} to calculate distances
on the WordNet graph. JavaSimLibrary provides implementations of the Lin and Jiang \& Conrath measures.

We implemented three different matchers based on WordNet:
\begin{itemize}
    \item WordNetDistanceMatching
    \item LinSimilarityMatching
    \item SynonymMatching
\end{itemize} 

The \textit{WordNetDistanceMatching} recognizer uses the Jiang and Conrath distance and calculates the average distance
between all the words in a sentence, normalized with the hypothesis size.

The \textit{LinSimilarityMatching} uses the Lin similarity measurement. In contrast to the
\textit{WordNetDistanceMatching} the \textit{LinSimilarityMatching} does not calculate averages, instead it is an
implementation of the \textit{BasicMatcher} class, and returns whether two nodes in the graph match by checking if there
similarity is bigger than 0.8. This value was found by trying different values.

The \textit{SynonymMatching} is also a \textit{BasicMatcher} that checks whether two graph nodes have a overlap in synonyms
with 1 graph level depth.
